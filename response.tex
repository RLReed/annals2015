\documentclass[a4paper,10pt]{article}
\usepackage[utf8]{inputenc}

%opening
\title{Response to Reviews - ANUCENE-D-14-00630}
\author{R. Reed}

\begin{document}

\maketitle


\section*{Responses to Reviewer 1}

\begin{enumerate}
    \item General Comments
    \begin{enumerate}
        \item The references have been updated to be consistent.
        
        \item The introduction and conclusions have been expanded to show how 
        application of the KLT will fit into the use of ERMM. 
    \end{enumerate}
    
    \item Abstract
    \begin{enumerate}
        \item The abstract has been expanded to define snapshots.  Additionally 
        the abstract was edited for clarity, and to better introduce the KLT 
        method as applied to current methods.
    \end{enumerate}
    
    \item Keywords
    \begin{enumerate}
        \item Replaced Transform with Karhunen-Lo\'{e}ve Transform
    \end{enumerate}
    
    \item Introduction
    \begin{enumerate}
        \item Additional background was added to the introduction to identify a 
        drawback of current applications of ERMM and how the proposed method 
        can provide a solution.
    \end{enumerate}
    
    \item Section 2
    \begin{enumerate}
        \item  The summation terms have been corrected
        \item  Further citations have been added to the history of ERMM in 
        section 1
    \end{enumerate}
    
    \item Section 3
    \begin{enumerate}
        \item Reworded the sentence to reference the work by Rahnema, Forget et 
        al.  
        \item Included the additional references to the work of Zhu
    \end{enumerate}
    
    \item Section 4
    \begin{enumerate}
        \item The statement about finer refinement in angles or space was added 
        after describing the two test problem.
        \item The spatial mesh scheme was already stated in the article in 
        section 4 as ``step-characteristics spatial discretization were used 
        for all calculations in both problems.''
        \item The material enrichment was added to the test problem description 
        for the 10-pin problem.
        \item Reworded the sentence to correct for the tenses.
        \item The correction for boiling water reactor was made.
        \item The reference to the work of Nichita et al. was included.  Our 
        test problem is not the same as that in their work, but our was adapted 
        from the work of Nichita.
        \item The choice of using SC in our work was arbitrary, but in previous 
        efforts, DD was used.  ERMM is insensitive to the underlying spatial 
        discretization.
    \end{enumerate}
    
    \item Section 5
    \begin{enumerate}
        \item No, a zeroth order expansion does not correspond to a coarse 
        group flux weighted energy condensation.  ERMM is nonlinear, so there is 
        no way to preserve fine-group reaction rates.  Although this is an 
        interesting theoretical point, a discussion of energy condensation 
        methods (i.e., the processes by which fine-group data is turned into 
        few-group data) is outside our scope of the present work.
        
        \item We have previously considered energy segmentation, but not the 
SPH factors.  Neither of these methods are discussed in the paper.

        \item The comparison of the spectrum from the 44-group and 238-group 
data was missing from the paper and has been appropriately treated.  The 
 44-group pin powers have an maximum error relative to the 238-group pin powers 
of over 0.5\%.  Since the accuracy goal is sub 0.1\% relative error, the 
238-group results provide significant improvement compared to the 44-group 
results.  This comparison was performed for the 10-pin problem as described in 
the present work.

        \item The wording for ``described methods described'' has been corrected
        
        \item ``Preformed'' was corrected
    \end{enumerate}
    
    \item Conclusion
    \begin{enumerate}
        \item The conclusion was extended similarly to the introduction and 
abstract to yield a better demonstration as to how our method of KLT is 
applicable to the broader use of ERMM and other similar methods.
    \end{enumerate}
\end{enumerate}

\section*{Additional Changes to the Article}

\begin{enumerate}
    \item Several grammar corrections were made throughout the paper.  Many of 
these include adding in missing articles.
    \item notation and wording has been standardize throughout the paper.
    \item Section 3.4.1 has been extended to show how the KLT basis functions 
are derived completely.
    \item Switched ``realistic'' to ``practical'' with regards to certain 
snapshot models.
    \item Clarified the wording in section 5.  The basis sets were better 
described as being formed from snapshots of the scalar flux and the leftward 
partial current.
    \item Removed figures of the BWR test problem results of including 
snapshots of the higher order angular moments.  The result is the same as for 
the 10-pin test problem, and the figures did not add meaningfully to the 
discussion.
    \item Added a section for acknowledgements
\end{enumerate}


\end{document}
